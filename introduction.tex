% Free range VHDL
% Authors: Bryan Mealy, Fabrizio Tappero
% Date: May, 2011
%
% (C) 2011 B. Mealy, F. Tappero
%
% !TEX root = master.tex
%
\phantomsection
\addcontentsline{toc}{chapter}{Purpose of this book} % add chapter*{Purpose of this book}
\chapter*{Purpose of this book}
The purpose of this book is to provide students and young engineers with a guide to help them develop the skills necessary to be able to use VHDL for introductory and intermediate level digital design. These skills will also give you the ability and the confidence to continue on with VHDL-based digital design. In this way, you will also take steps toward developing the skills required to implement more advanced digital design systems.
Although there are many books and on-line tutorials dealing with VHDL, these sources are often troublesome for several reasons. Firstly, much of the information regarding VHDL is either needlessly confusing or poorly written. Material with these characteristics seems to be written from the standpoint of someone who is either painfully intelligent or has forgotten that their audience may be seeing the material for the first time. Secondly, the common approach for most VHDL manuals is to introduce too many topics and a lot of extraneous information too early. Most of this material would best appear later in the presentation. Material presented in this manner has a tendency to be confusing, is easily forgotten if misunderstood or simply is never applied. The approach taken by this book is to provide only what you need to know to quickly get up and running in VHDL. As with all learning, once you have obtained and applied some useful information, it is much easier to build on what you know as opposed to continually adding information that is not directly applicable to the subjects at hand.

The intent of this book is to present topics to someone familiar with digital logic design and with some skills in algorithmic programming languages such as Java or C. The information presented here is focused on giving a solid knowledge of the approach and function of VHDL. With a logical and intelligent introduction to basic VHDL concepts, you should be able to quickly and efficiently create useful VHDL code. In this way, you will see VHDL as a valuable design, simulation and test tool rather than another batch of throw-away technical knowledge encountered in some forgotten class or lab.

Lastly, VHDL is an extremely powerful tool. The more you understand as you study and work with VHDL, the more it will enhance your learning experience independently of your particular area of interest. It is well worth noting that VHDL and other similar hardware design languages are used to create most of the digital integrated circuits found in the various electronic gizmos that overwhelm our modern lives. The concept of using software to design hardware that is controlled by software will surely provide you with endless hours of contemplation. VHDL is a very exciting language and mastering it will allow you to implement systems capable of handling and processing in parallel ns-level logic events in a comfortable software environment.

This book was written with the intention of being freely available to everybody. The formatted electronic version of this book is available from the Internet. Any part of this book can be copied, distributed and modified in accordance with the conditions of its license.
\vspace{10pt}

\noindent
\textbf{DISCLAIMER:}
This book quickly takes you down the path toward understanding VHDL and writing solid VHDL code. The ideas presented herein represent the core knowledge you will need to get up and running with VHDL. This book in no way presents a complete description of the VHDL language. In an effort to expedite the learning process, some of the finer details of VHDL have been omitted from this book. Anyone who has the time and inclination should feel free to further explore the true depth of the VHDL language. There are many on-line VHDL reference books and free tutorials. If you find yourself becoming curious about what this book is not telling you about VHDL, take a look at some of these references.

